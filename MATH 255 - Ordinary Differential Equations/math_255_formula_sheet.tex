\documentclass[10pt]{article}
%----------Packages----------
\usepackage[utf8]{inputenc}
\usepackage{amsmath}
\usepackage{amssymb}
\usepackage{multicol}
\usepackage[landscape, total={10.8in,8in}]{geometry}
\usepackage{blindtext}
\usepackage{graphicx}
\usepackage{tikz}
\usetikzlibrary{arrows.meta}
\usepackage{enumitem}
\usepackage{array}
%----------Page formatting----------
\pagenumbering{gobble}
\parindent=0pt
%----------Symbols----------
\newcommand{\C}{{\mathbb C}}
\newcommand{\N}{{\mathbb N}}
\newcommand{\R}{{\mathbb R}}
\newcommand{\Z}{{\mathbb Z}}
\newcommand{\ep}{\varepsilon}
\newcommand{\lap}[1]{\mathcal{L}\left\{#1\right\}}
\newcommand{\lapinv}[1]{\mathcal{L}^{-1}\left\{#1\right\}}
\newcommand{\ihat}{\hat{i}}
\newcommand{\jhat}{\hat{j}}
\newcommand{\khat}{\hat{k}}
%----------Vectors----------
\newcommand{\proj}{\mathrm{proj}}
\newcommand{\rank}{\mathrm{rank}}
\newcommand\norm[1]{\lVert#1\rVert}
\newcommand\vect[1]{\overrightarrow{#1}}
%----------Matrices----------
\newenvironment{amatrix}[1]{\left[\begin{array}{@{}*{#1}{c}|c@{}}}{\end{array}\right]}
\newcommand{\matr}[1]{\mathbf{#1}}
\newcommand{\tran}{^{\mkern-1.5mu\mathsf{T}}}
\newcommand{\conj}[1]{\overline{#1}}
\newcommand{\inv}{^{-1}}

\newcommand{\vvec}[2]{\begin{bmatrix}#1 \\ #2 \end{bmatrix}}
\newcommand{\vvvec}[3]{\begin{bmatrix}#1 \\ #2 \\ #3 \end{bmatrix}}
%----------Complex Numbers----------
\newcommand{\cis}{\mathrm{cis }}

%----------Headings----------
\newcommand\sectionheading[1]{\begin{center}\large{\textbf{#1}}\end{center}\normalsize}
\newcommand\heading[1]{\smallskip\textbf{#1}\smallskip}
% \newcommand\heading[1]{\textbf{#1}}
% Section Box, taken from https://www.overleaf.com/articles/130-cheat-sheet/ntwtkmpxmgrp
\tikzstyle{mybox} = [draw=black, fill=white, very thick,
    rectangle, rounded corners, inner sep=10pt, inner ysep=10pt]
\tikzstyle{fancytitle} =[fill=black, text=white, font=\bfseries]
\newenvironment{sectionbox}[2]{
    \begin{tikzpicture}
    % Content
    \node [mybox] (box){\begin{minipage}{0.294\textwidth}{#2}\end{minipage}};
    % Title
    \node[fancytitle, right=10pt] at (box.north west) {#1};
    \end{tikzpicture}
}

%----------Phase Diagrams----------
\newcommand*{\TickSize}{4pt}
\newcommand*{\DrawVerticalPhaseLine}[7][]{
\begin{center}
\begin{tikzpicture}[thick]
    % #1 = axis label
    % #2 = axis tick positions
    % #3 = axis tick labels
    % #4 = up arrows positions as CSV
    % #5 = down arrow positions as CSV
    % #6 = lower bound
    % #7 = upper bound
    \def\ArrowSpace{0.15}
    \foreach \X [count=\Xi] in {#2} {
    \foreach \Y [count=\Yi] in {#3} {
        \ifnum\Xi=\Yi 
            \draw (-\TickSize,\X) -- (\TickSize,\X) node [right] {\Y};
        \fi
    }}
    \foreach \X in {#4} { % Up arrows
        \draw [-Triangle] (0,\X-\ArrowSpace) -- (0,\X+\ArrowSpace);
    }
    \foreach \X in {#5} { % Down arrows
        \draw [-Triangle] (0,\X+\ArrowSpace) -- (0,\X-\ArrowSpace);
    }
    \draw (0,#6) -- (0,#7) node [above] {#1};
\end{tikzpicture}
\end{center}}

%----------Info----------
\newcommand*{\course}{MATH 255} 

%----------Document Begins Here----------
\begin{document}

\begin{center}
    \huge{\textbf{\course \ Formula Sheet}}
\end{center}

\begin{multicols*}{3}

\sectionheading{Definitions}

\heading{Differential Equation}

\[y^{(n)}=F(t,y,y',\ldots,y^{(n-1)})\]

\heading{Ordinary Differential Equation}

Derivatives are taken w.r.t. the same independent variable. 

\heading{Order}

The order of the highest derivative that appears.

\heading{Linear Differential Equation}

\[y^{(n)}+a_{n-1}y^{(n-1)}+\cdots+a_1y'+a_0 y=f(t)\]

Only linear combination of unknown and its derivatives.

\sectionheading{First Order ODEs}
\[y'(t)=f(t,y(t))\]

\heading{Integration: $y'(t)=f(t)$}
\[y(t)=\int f(t)\,dt\]

\heading{Separable Equations: $y'(t)=g(t)h(y)$}
\[\int\frac{dy}{h(t)}=\int g(t)\,dt\]

\heading{Integrating Factor: $y'+p(t)y=q(t)$ (linear)}
\[r(t)=e^{\int p(t)\,dt}\]
\[\bigl[r(t)y\bigr]'=r(t)q(t)\]
\[y=\frac{1}{r(t)}\Bigl(\int r(t)q(t)\,dt\Bigr)\]

\heading{Picard's Theorem: $y'=f(t,y)$.}

If $f(t,y)$ is continuous near $(t_0,y_0)$ and $\frac{\partial f}{\partial y}$ exists and is continuous near $(t_0,y_0)$, then a solution exists locally and is unique.

\heading{Autonomous Equation: $y'=f(y)$}

Critical points at $y'=f(y)=0$.

\begin{multicols*}{3}
    \DrawVerticalPhaseLine[$y$]{0}{Stable}{-0.3}{0.3}{-0.5}{0.5}
    \DrawVerticalPhaseLine[$y$]{0}{Unstable}{0.3}{-0.3}{-0.5}{0.5}
    \DrawVerticalPhaseLine[$y$]{0}{Semistable}{-0.3,0.3}{}{-0.5}{0.5}
\end{multicols*}

\heading{Application: Mixing}

$Q$: amount of salt

$s$: inflow concentration 

$r$: flow rate

$V$: volume

\[Q'=rs-\frac{r}{V}Q\]
\[Q(t)=sV+(Q_0-sV)e^{-\frac rVt}\]

\heading{Euler's Method: $y'=f(t,y)$}
\[y(t+h)\approx y(t)+hf(t,y)\]

\sectionheading{Second Order Linear ODEs}
\[y''+p(t)y'+q(t)y=g(t)\]
\[L[y]=g(t)\]

\heading{Homogeneous: $L[y]=0$}

If $y_1$ and $y_2$ are solutions, then $y=C_1y_1+C_2y_2$ is also a solution.

\heading{Reduction of Order: $y_1$ solves $y''+p(t)y'+q(t)y=0$}

Try $y_2(t)=v(t)y_1(t)$.
\[y_2(t)=y_1(t)\int\frac{e^{-\int p(t)\,dt}}{\bigl(y_1(t)\bigr)^2}\,dt\]

\heading{Constant Coefficient 2LODE: $ay''+by'+cy=0$}

Try $y=e^{rt}$:
\[ar^2+br+c=0\]
\begin{align*}
    y&=C_1e^{r_1x}+C_2e^{r_2x} &r_1\neq r_2 \\
    y&=(C_1+C_2x)e^{r_x} &r_1=r_2\\
    y&=C_1e^{\alpha x}\cos(\beta x)+C_2e^{\alpha x}\sin(\beta x) &r_1,r_2=\alpha\pm i\beta
\end{align*}

\heading{Nonhogeneous: $L[y]=g(t)\ne 0$}

\[y=y_c+y_p\]
where $y_c$ solves the homogeneous equation and $y_p$ is a particular solution.

\heading{Undetermined Coefficients: $L[y]=g(t)$}

$s=0,1,$ or $2$ so $y_c$ and $y_p$ are linearly independent.

\begin{tabular}{|c|c|}
    \hline
    $g(t)$ & $y_p$ \\
    \hline
    $P_n(t)=a_0t^n+\cdots+a_n$ & $t^s(A_0 t^n+\cdots+A_n)$ \\
    $e^{\alpha t}$ & $t^s Ae^{\alpha t}$ \\
    $\sin(\beta t)$ or $\cos(\beta t)$ & $A\cos(\beta t)+B\sin(\beta t)$ \\
    \hline 
\end{tabular}

\heading{Variation of Parameters: $L[y]=g(t)$}

Try $y_p=u_1 y_1 + u_2 y_2$.

\[\begin{cases}
    u_1' y_1 + u_2' y_2 = 0 \\
    u_1' y_1' + u_2' y_2' = g(t)
\end{cases}\]

\sectionheading{Applications of 2LODE}

\heading{Mechanical Oscillations}

\[mx''+cx'+kx=F(t), \quad\omega_0=\sqrt{\frac km}\]

% LHS: treat as constant coefficient 2LODE.
Undamped forced: $mx''+kx=F_0\cos(\omega t)$:
\[x_c=C_1\cos(\omega t)+C_2\sin(\omega t)\]
\[x_p=\frac{F_0}{m(\omega_0^2-\omega^2)}\cos(\omega t) \quad (\omega\ne\omega_0)\]
\[x_p=\frac{F_0}{2m\omega}t\sin(\omega t) \quad (\omega=\omega_0)\]

Damped forced: $mx''+cx'+kx=F_0\cos(\omega t)$:

Let $\rho=\frac{c}{2m}$. Call $x_p=x_{sp}$ (steady periodic).

Practical resonance: maximum amplitude of $x_{sp}$. 
 
\[\omega=\sqrt{\omega^2-2\rho^2} \text{ or } \omega=0\]

\sectionheading{Laplace Transform}

\heading{Definition}

\[\lap{f(t)}=F(s):=\int_0^\infty e^{-st}f(t)\,dt\]

$\lap{f+g}=\lap{f}+\lap{g} \text{ and }\lap{cf}=c\lap{f}$

$\lap{fg}\ne\lap{f}\lap{g}$

\newpage

\heading{Common Laplace Transforms}

\begin{tabular}{|c|c|}
    \hline
    $f(t)$ & $\lap{f(t)}=F(s)$ \\
    \hline 
    $c$ & $\frac cs, \quad s>0$ \\[5pt]
    $t^n$ & $\frac{n!}{s^{n+1}}, \quad s>0$ \\[5pt]
    $e^{-at}$ & $\frac{1}{s+a}, \quad s>-a$ \\[5pt]
    $\sin(at)$ & $\frac{a}{s^2+a^2}, \quad s>0$ \\[5pt]
    $\cos(at)$ & $\frac{s}{s^2+a^2}, \quad s>0$ \\[5pt]
    $\sinh(at)$ & $\frac{a}{s^2-a^2}, \quad s>|a|$ \\[5pt]
    $\cosh(at)$ & $\frac{s}{s^2-a^2}, \quad s>|a|$ \\[5pt]
    $e^{-at}f(t)$ & $F(s+a)$ \\[5pt]
    \hline
\end{tabular}

\heading{Laplace Transform of Derivatives}

\begin{tabular}{|c|l|}
    \hline 
    $f(t)$ & $\lap{f(t)}=F(s)$ \\
    \hline 
    $g'(t)$ & $s\lap{g(t)}-g(0)$ \\
    $g''(t)$ & $s^2\lap{g(t)}-sg(0)-g'(0)$ \\
    $g'''(t)$ & $s^3\lap{g(t)}-s^2g(0)-sg'(0)-g''(0)$ \\
    $g^{(n)}(t)$ & \footnotesize{$s^n\lap{g(t)}-s^{n-1}g(0)-s^{n-2}g'(0)-\cdots-g^{(n-1)}(0)$} \\
    \hline
\end{tabular}

\heading{Solving ODEs}

\begin{enumerate}[noitemsep,topsep=0pt]
    \item Laplace transform both sides 
    \item Plug in initial condition 
    \item Solve for $X(s)=\lap{x(t)}$
    \item $\lap{x(t)}=\lapinv{X(s)}$
\end{enumerate}

\heading{Convolution}
\[\lapinv{F(s) G(s)}=f*g=\int_0^t f(\tau)g(t-\tau)d\,\tau \quad (t\ge 0)\]

\sectionheading{Linear System of ODEs}
\[\vec{x}\,'=A(t)\vec x(t)+\vec f(t) \quad\Leftrightarrow\quad \vec x\,'=P\vec x+\vec f\]

\heading{Homogeneous}

\[\vec x=c_1\vec x_1+c_2\vec x_2+\cdots+c_n\vec x_n\]
\[X(t)=\begin{bmatrix}
    | & | & & | \\
    \vec x_1 & \vec x_2 & \cdots & \vec x_n \\
    | & | & & | 
\end{bmatrix} \text{ (fundamental matrix)}\]
\[\vec x=X\vec c\]

\heading{Nonhomogeneous}

\[\vec x=\vec x_c+\vec x_p=X\vec c+\vec x_p\]

\heading{Eigenvalue Method}

Eigenvalue: $\det(A-\lambda I)=0$

Eigenvector: $(A-\lambda I)\vec v=0$

Solution to $\vec x\,'=P\vec x$:
\[\vec x=c_1\vec v_1e^{\lambda_1 t}+c_2\vec v_2e^{\lambda_2 t}+\cdots+c_n\vec v_ne^{\lambda_n t}\]

If complex eigenpair $\vec x_1=\vec v_1e^{\lambda_1 t}$, $\vec x_2=\conj{\vec v_1}$: 
\begin{itemize}[noitemsep,topsep=0pt]
    \item Use $c_1 \Re(\vec x_1)+c_2 \Im(\vec x_1)$ 
\end{itemize}

\heading{Vector Field Behaviour}

\begin{tabular}{|c|c|c|c|}
    \hline
    $\lambda_1$ & $\lambda_2$ & behaviour & stability \\
    \hline
    $+$ & $+$ & source & unstable \\
    $-$ & $-$ & sink & stable \\
    $+$ & $-$ & saddle & unstable\\
    \hline
\end{tabular}

\begin{tabular}{|c|c|c|}
    \hline
    $\lambda$ & behaviour & stability \\
    \hline
    $\pm bi$ & center & ellipses \\
    $a\pm bi, a>0$ & spiral source & unstable \\
    $a\pm bi, a<0$ & spiral sink & stable \\
    \hline
\end{tabular}

\heading{Multiple Eigenvalues}

Generalized eigenvectors:
\begin{align*}
    (A-\lambda I)\vec v_1&=\vec v_0 \implies \vec x_1 = \vec v_1e^{\lambda t} \\
    (A-\lambda I)\vec v_2&=\vec v_1 \implies \vec x_2 = (\vec v_2+\vec v_1 t)e^{\lambda t} \\
    (A-\lambda I)\vec v_k&=\vec v_{k-1} \implies \vec x_k=\left(\sum_{i=0}^{k-1}\vec v_{k-i}\frac{t^i}{i!}\right)e^{\lambda t}
\end{align*}

\sectionheading{Nonlinear System of ODEs}

\heading{Autonomous systems}

\[\frac{d}{dt}\underbrace{\vvec{x(t)}{y(t)}}_{\vec x}=\underbrace{\vvec{f(x,y)}{g(x,y)}}_{F(\vec x)}\]

Critical points: $F(\vec x_0)=0$.

\heading{Linearization}

Suppose $\vec p=(x_0,y_0)$ is a critical point. 

Let $u=x-x_0$ and $v=y-y_0$. 

Linearized system:
\[\vvec{u}{v}'=\matr{J}_{\vec{p}}\vvec{u}{v}=\begin{bmatrix}
    f_x & f_y \\
    g_x & g_y
\end{bmatrix}_{\vec p}\vvec{u}{v}\]

"Almost linear" - good approximation when:
\begin{itemize}[noitemsep,topsep=0pt]
    \item $(x_0,y_0)$ is isolated
    \item Jacobian is invertible $\Leftrightarrow$ $0$ is not an eigenvalue
\end{itemize}

\heading{Classification}

Refer to Vector Field Behaviour. Center behaviour is unclear because Jacobian of nonlinear system will vary.

\heading{Conservative Equation}

\[x''+f(x)=0\]

Associated conservative system:
\[\vvec{x}{y}'=\vvec{y}{-f(x)}\]

Energy function $E(x,y)$:
\[E(x,y)=\frac 12y^2+\int_0^x f(u)\,du\]

For any solution, $E(x,y)$ is constant.

\sectionheading{Miscellaneous Formulas}

\heading{Partial Fractions}

{\footnotesize
\[\frac{P(x)}{(x-r_1)(x-r_2)\cdots(x-r_n)}=\frac{A_1}{x-r_1}+\frac{A_2}{x-r_2}+\cdots+\frac{A_n}{x-r_n}\]}

$(x-r)^m$ corresponds to $\displaystyle\frac{A_1}{x-r}+\frac{A_2}{(x-r)^2}+\cdots+\frac{A_m}{(x-r)^m}$

$ax^2+bx+c$ corresponds to $\frac{Ax+B}{ax^2+bx+c}$

\sectionheading{Identities and Derivatives}
\heading{Trig}
\[\cos(2x)=\cos^2 x-\sin^2 x \quad \sin(2x)=2\sin x\cos x\]
\[\sin(x\pm y)=\sin x \cos y \pm \cos x \sin y\]
\[\cos(x\pm y)=\cos x \cos y \mp \sin x \sin y\]
\[\tan(x\pm y)=\frac{\tan x \pm \tan y}{1 \mp \tan x \tan y}\]
\heading{Derivatives} 
\begin{multicols*}{2}
\[\frac{d}{dx} (\tan x)=\sec^2 x\]
\[\frac{d}{dx} (\csc x)=-\csc x \cot x\]
\[\frac{d}{dx} (\sec x)=\sec x \tan x\]
\[\frac{d}{dx} (\cot x)=-\csc^2 x\]
\[\frac{d}{dx} (\arcsin x)=\frac{1}{\sqrt{1-x^2}}\]
\[\frac{d}{dx} (\arccos x)=-\frac{1}{\sqrt{1-x^2}}\]
\[\frac{d}{dx} (\arctan x)=\frac{1}{1+x^2}\]
\[\frac{d}{dx} \log |x| = \frac{1}{x}\]
\[\frac{d}{dx} b^x = b^x \log b\]
\[\frac{d}{dx} \log_b x=\frac{1}{x\log b}\]
\end{multicols*}

\end{multicols*}

\end{document}