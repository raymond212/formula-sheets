\documentclass[10pt]{article}
%----------Packages----------
\usepackage[utf8]{inputenc}
\usepackage{amsmath}
\usepackage{amssymb}
\usepackage{multicol}
\usepackage[landscape, total={10.8in,8in}]{geometry}
\usepackage{blindtext}
\usepackage{graphicx}
\usepackage{tikz}
\usetikzlibrary{arrows.meta}
\usepackage{enumitem}
\usepackage{array}
\usepackage{bm}
%----------Page formatting----------
\pagenumbering{gobble}
\parindent=0pt
%----------Symbols----------
\newcommand{\C}{{\mathbb C}}
\newcommand{\N}{{\mathbb N}}
\newcommand{\R}{{\mathbb R}}
\newcommand{\Z}{{\mathbb Z}}
\newcommand{\ep}{\varepsilon}
\newcommand{\lap}[1]{\mathcal{L}\left\{#1\right\}}
\newcommand{\lapinv}[1]{\mathcal{L}^{-1}\left\{#1\right\}}
\newcommand{\ihat}{\hat{i}}
\newcommand{\jhat}{\hat{j}}
\newcommand{\khat}{\hat{k}}
%----------Vectors----------
\newcommand{\proj}{\mathrm{proj}}
\newcommand{\rank}{\mathrm{rank}}
\newcommand{\norm}[1]{\lVert#1\rVert}
\newcommand{\vect}[1]{\overrightarrow{#1}}
\newcommand{\bvec}[1]{\bm{\textnormal{\bfseries #1}}}
%----------Matrices----------
\newenvironment{amatrix}[1]{\left[\begin{array}{@{}*{#1}{c}|c@{}}}{\end{array}\right]}
\newcommand{\matr}[1]{\mathbf{#1}}
\newcommand{\tran}{^{\mkern-1.5mu\mathsf{T}}}
\newcommand{\conj}[1]{\overline{#1}}
\newcommand{\inv}{^{-1}}

\newcommand{\vvec}[2]{\begin{bmatrix}#1 \\ #2 \end{bmatrix}}
\newcommand{\vvvec}[3]{\begin{bmatrix}#1 \\ #2 \\ #3 \end{bmatrix}}
%----------Complex Numbers----------
\newcommand{\cis}{\mathrm{cis }}
%----------Derivative----------
\newcommand{\dv}[2]{\frac{\mathrm{d} #1}{\mathrm{d} #2}}
\newcommand{\ddv}[2]{\frac{\mathrm{d}^2 #1}{\mathrm{d} #2^2}}
\newcommand{\pd}[2]{\frac{\partial #1}{\partial #2}}
\newcommand{\pdd}[2]{\frac{\partial^2 #1}{\partial #2^2}}

%----------Headings----------
\newcommand\sectionheading[1]{\begin{center}\large{\textbf{#1}}\end{center}\normalsize}
\newcommand\heading[1]{\smallskip\textbf{#1}\smallskip}
% \newcommand\heading[1]{\textbf{#1}}

%----------Info----------
\newcommand*{\course}{MATH 253} 

%----------Document Begins Here----------
\begin{document}

\begin{center}
    \huge{\textbf{\course \ Formula Sheet}}
\end{center}

\begin{multicols*}{3}

\sectionheading{Vectors}

\sectionheading{Lines and Planes}

\sectionheading{Limits}
\begin{tabular}{@{}ll}
    Limit & $\displaystyle\lim_{\bvec x\to\bvec a}f(\bvec x)=L$ \\
    Continuity & $\displaystyle\lim_{\bvec x\to\bvec a}f(\bvec x)=f(\bvec a)$
\end{tabular}
where $\bvec x\to\bvec a$ in any way.

\heading{Showing limit DNE}

\begin{itemize}[itemsep=0pt,topsep=0pt]
    \item Find one way to approach $\bvec a$ such that the limit DNE.
    \item If $\bvec a=(0,0)$, we could test $\displaystyle\lim_{t\to 0}f(t,0)$, $\displaystyle\lim_{t\to 0}f(0,t)$, $\displaystyle\lim_{t\to 0}f(t,t)$, etc.
\end{itemize}

\heading{Showing limit exists}

\begin{itemize}[itemsep=0pt,topsep=0pt]
    \item Use polar coordinates, then take $r\to 0$. $\theta$ does not matter.
    \item i.e. Set $x=r\cos\theta,y=r\sin\theta$, then $x^2+y^2=r^2$.
\end{itemize}

\sectionheading{Partial Derivatives}

\heading{First order}
\[f_x=f_1=\pd fx=\partial_x f=\lim_{h\to 0}\frac{f(x+h,y)-f(x,y)}{h}\]
\[f_y=f_2=\pd fy=\partial_y f=\lim_{h\to 0}\frac{f(x,y+h)-f(x,y)}{h}\]

\heading{Higher order}
\begin{align*}
    f_{xx}&=f_{11}=\pdd fx \\
    f_{yy}&=f_{22}=\pdd fy \\
    f_{xy}&=f_{12}=\pd fy\left(\pd fx\right) \\
    f_{yx}&=f_{21}=\pd fx\left(\pd fy\right)
\end{align*}
{\small If $f_{xy}$ and $f_{yx}$ both exist and are continuous, then $f_{xy}=f_{yx}$.}

\heading{Chain Rule}

Consider $f(x,y)$ with $x(t)$ and $y(t)$.
\[\dv ft=\pd fx\dv xy+\pd fy\dv yt\]

Consider $f(x,y)$ with $x(s,t)$ and $y(s,t)$.
\[\pd ft=\pd fx\pd xt+\pd fy\pd yt\]

Consider $f(x_1,x_2,\ldots,x_n)$ with $x_i(t_1,t_2,\ldots,t_k)$.
\[\pd f{t_i}=\pd{f}{x_1}\pd{x_1}{t_i}+\pd{f}{x_2}\pd{x_2}{t_i}+\cdots+\pd{f}{x_n}\pd{x_n}{t_i}\]

\sectionheading{Tangent Planes}

Tangent plane at $f(a,b)$:
\[-f_x(a,b)(x-a)-f_y(a,b)(y-b)+(z-f(a,b))=0\]

\sectionheading{Gradient}

Consider $f(x,y,z)$. 
\[\nabla f=\bigg\langle \pd fx,\pd fy,\pd fz\bigg\rangle\]

\begin{itemize}[noitemsep,topsep=0pt]
    \item $\nabla f(a,b,c)$ is the direction of normal vector at $(a,b,c)$ 
    \item Point-normal tangent plane:
    \[\nabla f(a,b,c)\cdot\langle x-a,y-b,z-c\rangle=0\]
    \item Normal line:
    \[\langle x,y,z\rangle=\langle a,b,c\rangle+t\cdot\nabla f(a,b,c)\]
\end{itemize}

\sectionheading{Linear Approximation}
\[f(x,y)\approx f(a,b)+f_x(a,b)(x-a)+f_y(a,b)(y-b)\]
\[f(x+\Delta x,y+\Delta y)\approx f(a,b)+f_x(a,b)\Delta x+f_y(a,b)\Delta y\]
\[\Delta f\approx f_x(a,b)\Delta x+f_y(a,b)\Delta y\]
\[\mathrm{d}f=f_x(a,b)\mathrm{d}x+f_y(a,b)\mathrm{d}y\]

\end{multicols*}

\end{document}